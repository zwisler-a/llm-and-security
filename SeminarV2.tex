\documentclass{SeminarV2}
\usepackage[dvips]{graphicx}
\usepackage[latin1]{inputenc}
\usepackage{amssymb,amsmath,array}

%***********************************************************************
% !!!! IMPORTANT NOTICE ON TEXT MARGINS !!!!!
%***********************************************************************
%
% Please avoid using DVI2PDF or PS2PDF converters: some undesired
% shifting/scaling may occur when using these programs
% It is strongly recommended to use the DVIPS converters.
%
% Check that you have set the paper size to A4 (and NOT to letter) in your
% dvi2ps converter, in Adobe Acrobat if you use it, and in any printer driver
% that you could use.  You also have to disable the 'scale to fit paper' option
% of your printer driver.
%
% In any case, please check carefully that the final size of the top and
% bottom margins is 5.2 cm and of the left and right margins is 4.4 cm.
% It is your responsibility to verify this important requirement.  If these margin requirements and not fulfilled at the end of your file generation process, please use the following commands to correct them.  Otherwise, please do not modify these commands.
%
\voffset 0 cm \hoffset 0 cm \addtolength{\textwidth}{0cm}
\addtolength{\textheight}{0cm}\addtolength{\leftmargin}{0cm}

%***********************************************************************
% !!!! USE OF THE SeminarV2 LaTeX STYLE FILE !!!!!
%***********************************************************************
%
% Some commands are inserted in the following .tex example file.  Therefore to
% set up your Seminar submission, please use this file and modify it to insert
% your text, rather than staring from a blank .tex file.  In this way, you will
% have the commands inserted in the right place.

% Edited by Martin Bogdan.

\begin{document}
%style file for Seminar manuscripts
\title{Attack vectors and security risks of large language models}

%***********************************************************************
% AUTHORS INFORMATION AREA
%***********************************************************************
\author{Alexander Zwisler$^1$
%
% DO NOT MODIFY THE FOLLOWING '\vspace' ARGUMENT
\vspace{.3cm}\\
%
% Addresses and institutions (remove "1- " in case of a single institution)
1- University Leipzig - Faculty for Mathematics and Computer Science \\
}
%***********************************************************************
% END OF AUTHORS INFORMATION AREA
%***********************************************************************

\maketitle

\begin{abstract}
Type your 100 words abstract here. Please do not modify the style
of the paper. In particular, keep the text offsets to zero when
possible (see above in this `SeminarV2.tex' sample file). You may
\emph{slightly} modify it when necessary, but strictly respecting
the margin requirements is mandatory (see the instructions to
authors for more details).
\end{abstract}

\section{Introduction}

Large language models (LLMs) are currently transforming the way we interact
with computer systems. With their unseen capabilities in natural language 
processing tasks they are a valueable system to integrate with a lot of 
existing systems. We currently see a fast paced integration of LLMs with 
pretty much all systems.

\subsection{Large language models}
What are large language models


\section{Prompt Injections}
This SeminarV2.tex file defines how to insert references, both for
BiBTeX and non-BiBTeX users.  Please read the instructions in this
file.

\section{LLM integrated systems}
This SeminarV2.tex file defines how to insert references, both for
BiBTeX and non-BiBTeX users.  Please read the instructions in this
file.

\subsection{Indirect Prompt Injections}


\section{Conclusion}

% ****************************************************************************
% BIBLIOGRAPHY AREA
% ****************************************************************************

\begin{footnotesize}

% IF YOU DO NOT USE BIBTEX, USE THE FOLLOWING SAMPLE SCHEME FOR THE REFERENCES
% ----------------------------------------------------------------------------
\begin{thebibliography}{99}

% For books
\bibitem{Haykin_book} S. Haykin, editor. \emph{Unsupervised Adaptive Filtering vol.1 : Blind Source Separation}, John Willey ans Sons, New York, 2000.

% For articles
\bibitem{DelfosseLoubaton_article}N. Delfosse and P. Loubaton, Adaptibe blind separation of sources: A deflation
approach, \emph{Signal Processing}, 45:59-83, Elsevier, 1995.


\end{thebibliography}
% ----------------------------------------------------------------------------

% IF YOU USE BIBTEX,
% - DELETE THE TEXT BETWEEN THE TWO ABOVE DASHED LINES
% - UNCOMMENT THE NEXT TWO LINES AND REPLACE 'Name_Of_Your_BibFile'

%\bibliographystyle{unsrt}
%\bibliography{Name_Of_Your_BibFile}

\end{footnotesize}

% ****************************************************************************
% END OF BIBLIOGRAPHY AREA
% ****************************************************************************

\end{document}
